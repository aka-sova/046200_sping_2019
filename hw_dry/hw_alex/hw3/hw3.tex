\documentclass[a4paper]{iacas}

\usepackage{cite}
\usepackage{hyperref}% embedding hyperlinks [must be loaded after dropping]
\usepackage{amsmath,amsthm,amssymb,amsfonts,latexsym,mathrsfs,wasysym}
\usepackage{marvosym}
\usepackage{subcaption}
\usepackage{soul,color}
\usepackage{threeparttable}% tables with footnotes
\usepackage{dcolumn}% decimal-aligned tabular math columns
\usepackage{float}
\usepackage{graphicx}
%\usepackage{accents}
\usepackage{tikz}
\usepackage{lastpage}
\usepackage{fancyhdr}
\usepackage{color}
\usepackage{cancel}
\usepackage{setspace}




%\doublespacing
% or:
\onehalfspacing
%\usepackage[T1]{fontenc}
%\usepackage{bigfoot} % to allow verbatim in footnote
\usepackage[framed,numbered]{matlab-prettifier}
\pagestyle{plain}
%\usepackage[hebrew,english]{babel}
\usetikzlibrary{shapes.geometric, arrows, calc}

\newcolumntype{d}{D{.}{.}{-1}}
\graphicspath{{figures/}}

% define some commands to maintain consistency
\newcommand{\pkg}[1]{\texttt{#1}}
\newcommand{\cls}[1]{\textsf{#1}}
\newcommand{\file}[1]{\texttt{#1}}
\newcommand{\sgn}[1]{\operatorname{sgn}\left(#1\right)}
\newcommand{\sat}[1]{\operatorname{sat}\left(#1\right)}
\newcommand{\rrule}[1]{\rule[#1]{0pt}{0pt}}
\newcommand{\fracds}[2]{\frac{\displaystyle #1\rrule{-0.2em}}{\displaystyle #2\rrule{1em}}}
\newcommand{\figref}[1]{Fig.~\ref{#1}}
\newcommand{\ubar}[1]{\underaccent{\bar}{#1}}
\newcommand{\norm}[1]{\lvert \lvert \vec #1 \rvert \rvert}

%diffeomorphism

% XeLaTeX!
% ============================================================ %
% HEBREW support via polyglossia %
% ============================================================ %
	\usepackage{polyglossia}
	\defaultfontfeatures{Mapping=tex-text, Scale=MatchLowercase}
	\setdefaultlanguage{english}
	\setotherlanguage{hebrew}
	\newfontfamily\hebrewfont[Script=Hebrew]{Times New Roman}
% Use \begin{hebrew} block of text \end{hebrew} for paragraphs.
% Use \texthebrew{ } and \textenglish{ } for short texts.
% ============================================================ %



\begin{document}

\begin{center}
 \large Image processing - 046200
 \end{center}
\begin{center}
\large\textbf{Homework \#3}
 \end{center}


\begin{tabular}{l}
\\
{\bf\textit{Alexander Shender 328626114}} \\
{\bf\textit{Sahar Carmel 305554453}} \\
Technion - Israel Institute of Technology
\end{tabular}

\vspace{2em}

\newpage


\begin{hebrew}
\textit{\huge שאלה 1.}

א. ניתן לראות שהפעולה שאנו מבצעים הינה פעולת קונבולוציה עם דלתא של הזזה שמוגדרת כ: 
\end{hebrew}
$\delta(m-m_i,n-n_i)$
\begin{hebrew}
ולכן אנו נוכל לרשום את התמונה של $X[m,n]$ כסכום של קונבולוציות:
\end{hebrew}
$$X[m,n]  =\sum^P_{x=1}\phi[m,n]*\delta(m-m_i,n-n_i) \Longrightarrow h[m,n] = \delta(m-m_i,n-n_i)$$

\begin{hebrew}

ב. הזזה הינה אכן פעולה ספרבילית, כי ניתן לפרק את התזוזה לציר $X$ וגם לציר $Y$. נרחיב את הביטוי שמצאנו, לפירוק לשני הצירים:
\end{hebrew}
$$X[m,n]  =\sum^P_{x=1}\phi[m,n]*\delta(m-m_i,n-n_i) = \sum^P_{x=1}\phi[m,n]*(\delta(m-m_i)\delta(n-n_i))$$
\begin{hebrew}
ג. בדרך כלל, כנגד ה-$salt \& pepper$ השיטה המועילה להתגבר עליה הינה המסנן החציון. אך במקרה שלנו זאת לא השיטה שתניב תוצאה סבירה:
\\בהנחה שהתמונות $\psi[m,n]$ מפוזרות מספיק רחב בתמונה אנו נקבל שמסנן חציון יאפס לנו את כל התמונה (אלא אם כן יצא מקרה דופק בו נקבל 5 פיקסלים עם ערך 1 בריבוע של 9 פיקסלים, מה שאינו סביר, כי רק $3\%$ מהפיקסלים הם מרועשים).
\\בהנחה שהתמונות $\psi[m,n]$ מפורזרות מאד צפוף, נקבל שפיקסלים שבמקור היו לבנים, יקבלו עכשיו ערכים. וזה לא רצוי.
\end{hebrew}
\newline
\begin{hebrew}
השיטה המועילה לדעתינו תהיה השיטה של $\textrm{template matching}$. כתבנית אנו נשתמש בתמונה $\psi[m,n]$ , ובמקום בו אנו נקבל התאמה מעל סף מסוים שנקבע, נכניס את התמונה  $\psi[m,n]$ בתמונה החדשה שנגדיר. ככה נקרב את התנונה $\hat{U}[m,n]$ לתמונה המקורית האמיתית $U[m,n]$ 
\end{hebrew}


\newpage
\begin{hebrew}
\textit{\huge שאלה 2.}

א. נמצא את הסינון הלינארי:
\end{hebrew}
\begin{equation*}
\psi = \alpha(1+k\nabla^2) = \alpha\left(\begin{bmatrix}0&0&0\\0&1&0\\0&0&0\end{bmatrix}+K\begin{bmatrix}0&1&0\\1&-4&1\\0&1&0\end{bmatrix}\right) = \begin{bmatrix}0&K&0\\K&\alpha-4K&K\\0&K&0\end{bmatrix}
\end{equation*}


\begin{hebrew}
ב. נמצא את $\alpha$ שישמור על הממוצע של התמונה המקורית. לטובת זאת, נשווה את הממוצע של הפילטר ל-1:
\end{hebrew}

\begin{equation*}
mean = 4\cdot K + \alpha - 4K = 1  \Longrightarrow  \alpha = 1
\end{equation*}

\begin{hebrew}
ג. כעת נתון ש-$\alpha$ הינו 1. כמו כן, מצאנו שגרעין $\psi$ הינו $3X3$, כלומר $M=N=3$. נמצא את K אשר ימזער את השגיאה הריבועית המוגדרת:
\end{hebrew}
\begin{equation*}
E = \sum_{m=1}^M \sum_{m=1}^M |\psi -\delta|^2
\end{equation*}














\end{document}





